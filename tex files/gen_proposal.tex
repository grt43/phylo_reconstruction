\documentclass[12pt]{article}
\usepackage[margin=1.25in]{geometry}
\usepackage{grt}
\usepackage{changepage}
\usepackage{hyperref}

\begin{document}
\gtitle
    {BTRY 4840: Project Proposal}
    {Garrett Tetrault, grt43}
    {\today}
\sectionline

For the final project, I would like to focus on the following paper by Felsenstein and Churchill
that presents a model to compute the likelihood of a phylogeny,
allowing for unequal evolutionary rates at different sites in the molecular sequences.
This paper was noted in Durbin's text in chapter eight 
when the discussion turns to creating a more realistic model of evolution.
\begin{adjustwidth}{1in}{1in}
    J Felsenstein, G A Churchill, 
    A Hidden Markov Model approach to variation among sites in rate of evolution., 
    Molecular Biology and Evolution, Volume 13, Issue 1, Jan 1996, Pages 93–104, \\
    \url{https://doi.org/10.1093/oxfordjournals.molbev.a025575}
\end{adjustwidth}
The underpinning of the model is a Hidden Markov Model whose states represent
different rates of genetic mutation for a site.
This model requires a prior distribution of rates and transition probabilities between rates.
The project would be divided into two main sections.
\begin{enumerate}
    \item The first would be to implement the algorithm to compute a phylogeny's likelihood
    and compute a maximum likelihood phylogeny from this.
    The paper describes an algorithm to do this using the above model to compute likelihoods 
    and estimating branch length by the Newton-Raphson method.

    \item The next portion of the project would be to consider different prior distributions of rates.
    For example, the paper points out that 
    a ``discrete distribution with four well-chosen classes'' preformed well.
    Alternatively, I would like to try some variation of 
    the truncated geometric distribution we encountered in our homework.
    Additionally, different amounts of hidden states could shed light on the effects of over-fitting.
\end{enumerate}

Possible data sets for the project would be that which was assigned in Problem Set 4,
or that which the paper detailed.
\end{document}